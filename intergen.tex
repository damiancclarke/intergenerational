\documentclass[11pt]{article}

\usepackage{amsmath}
\usepackage{appendix}
\usepackage{bm}
\usepackage{booktabs}
\usepackage[capposition=top]{floatrow}
\usepackage{fontspec}
\usepackage[bottom]{footmisc}
\usepackage[margin=1in,footskip=0.25in]{geometry}
\usepackage{graphicx}
\usepackage{lscape}
\usepackage{natbib}
\usepackage{setspace}
\usepackage{subcaption}
\usepackage{subfloat}
\bibliographystyle{abbrvnat}\bibpunct{(}{)}{;}{a}{,}{,}

%%%%
%%%%  NOTE: THIS MUST BE COMPILED WITH xelatex!
%%%%
\setmainfont{Times New Roman}
\setlength{\parskip}{6pt}


\title{The Long-Term and Intergenerational Effects of Neonatal Health Interventions\thanks{We are grateful to X, Y and Z for useful comments, and to seminar audiences at A, B and C. Clarke acknowledges financial support from ANID Chile (FONDECYT Regular 1200634).  OTHER.  }}
\author{Damian Clarke\thanks{Department of Economics, University of Chile \& IZA.  Contact: dclarke@fen.uchile.cl.}
  \and Nicol\'as Lillo Bustos\thanks{Department of Economics, University of Chile.  Contact:}
  \and Kathya Tapia Schythe\thanks{Department of Economics, University of California, Davis.  Contact:}
}
\date{\today}

\begin{document}
\begin{spacing}{1.5}
  \maketitle

  \begin{abstract}
    Targeted treatments of new-borns with delicate health stocks have been shown to have considerable returns in terms of survival and later life outcomes.  We seek to determine to what degree such treatments are reflected in later life health outcomes of treated individuals, and to what degree these are passed on to early life health outcomes of the \emph{subsequent} generation.  We follow three generations of linked microdata, and use a regression discontinuity design to study the impacts of targeted neonatal health policies based on birth weight assignment rules.  We estimate that...
  \end{abstract}

  \noindent JEL Codes: I11; I18; J13; H51; O15. \\
  Keywords: Early life interventions; intergenerational mobility; health care provision. \\
  %\end{spacing}

  \clearpage

  \section{Introduction}
  Returns to early life investment programs accrue over life.  For how long does this accrual last?  

  

  How long are returns to early life investment programs accrued?  Large public health programs focused on children with poor birth outcomes have been shown to result in immediate improvements in health and survival \citep{Almondetal2010}, which impact educational outcomes in childhood and adolescence \citep{Bharadwajetal2013}.  A broader literature suggests intergenerational transfers in education and well-being.  If neonatal and early life health programs spillover to future generations, this suggests that the already large benefits of such programs could be a (considerable) lower bound.


  ...
  
  We trace out impacts of an early life program through childhood, adolescence, and early adulthood, and additionally follow impacted individuals into the next generation.  Linking comprehensive microdata registries, we are able to observe all births in Chile between 1992-2017, their full inpatient hospitalization history, and for those that go on to have \emph{their own} births, we observe the early life health stocks and hospitalization records for their own children.  We study ...

  
  \clearpage
  \section{Background and Context}
  
  
  
  \clearpage
  \section{Data}
  Between 1992 and 2017 we observe x,xxx,xxx births occurring to x,xxx,xxx mothers.  Birth registries in Chile are universal, estimated to cover 99.9\% of all births [check cite DEIS].  Individuals are recorded using their national idendity number, assigned at birth. Data on these births are merged (using a masked version og the national identity number) with the hospitalization registry and the death registry, which cover all deaths and in-patient hospitalizations in the country.  In total, xx,xxx births are matched to a death record before the age of 1 year. This closely agrees with the average infant mortality rate reported over this time by the World Bank (which is 8.7 per 1,000 live births).\footnote{Note that there are x,xxx deaths which occur under 1 year of age in the death registry which are not observed in the birth registry. This will occur when a child arrives to the country after being born but does not survive up until 1 year.}  And in total xx,xxx births are observed to appear in the death registry at any point during this period.  Of all births, x,xxx,xxx are matched to at least one hospitalization, and in total xx,xxx,xxx hospitalizations are matched with births, implying that the average number of hospitalizations per matched birth is x.xx.  In Appendix \ref{app:data} we document tha match quality, observing that across registries individuals which are matched based on their unique national identity number are reported to have precisely the same birth date and sex.  In less than xx\% of the matches, we observe a reporting inconsistency.

  Of the births between 1992-2017, x,xxx,xxx (xx\%) are girls.  Of these, xx,xxx are observed to have \emph{their own} child in the birth registry.  This occurs when an individual who was born in an early year (say 1992) has a birth in a later year (say 2017, in which case she would be aged 25 years at the time of her own child's birth).  In Appendix \ref{app:data} (Table \ref{tab:birthChart}) we present the number of births from each year which went on to have a birth in the future.

  
  
  
  %Chile IMR (World Bank: https://data.worldbank.org/indicator/SP.DYN.IMRT.IN?locations=CL)
  %1992 - 13.7
  %1993 - 12.7
  %1994 - 11.8
  %1995 - 11.1
  %1996 - 10.6
  %1997 - 10.3
  %1998 - 10.1
  %1999 - 9.8
  %2000 - 9.2
  %2001 - 8.7
  %2002 - 8.3
  %2003 - 8.1
  %2004 - 7.9
  %2005 - 7.7
  %2006 - 7.6
  %2007 - 7.6
  %2008 - 7.6
  %2009 - 7.5
  %2010 - 7.4
  %2011 - 7.3
  %2012 - 7.2
  %2013 - 7.1
  %2014 - 6.9
  %2015 - 6.7
  %2016 - 6.6
  %2017 - 6.4
  
  

  
  \clearpage
  \section{Methods}
  \citet{Almondetal2010,Bharadwajetal2013}

  \citet{Calonicoetal2020a,Calonicoetal2014}
  
  \citet{Barrecaetal2011}

  \citet{RomanoWolf2005}
  
  \clearpage
  \section{Results}
  \subsection{The Early Life Impacts of Treatment on the First Generation}
  \begin{figure}[htpb!]
    \caption{Birthweight Assignment Thresholds and Infant Mortality}
    \label{fig:IMR}
    \begin{subfigure}{.45\textwidth}
      \centering
      %\includegraphics[width=.8\linewidth]{image_file_name}
      \caption{Infant Mortality $\geq$ 32 Weeks (BLN Method)}
      \label{fig:IMRBLN32}
    \end{subfigure}
    \begin{subfigure}{.45\textwidth}
      \centering
      %\includegraphics[width=.8\linewidth]{image_file_name}
      \caption{Infant Mortality $\leq$ 31 Weeks (BLN Method)}
      \label{fig:IMRBLN31}
    \end{subfigure}

    \begin{subfigure}{.45\textwidth}
      \centering
      %\includegraphics[width=.8\linewidth]{image_file_name}
      \caption{Infant Mortality $\geq$ 32 Weeks (Optimal)}
      \label{fig:IMROPT32}
    \end{subfigure}
    \begin{subfigure}{.45\textwidth}
      \centering
      %\includegraphics[width=.8\linewidth]{image_file_name}
      \caption{Infant Mortality $\leq$ 31 Weeks (Optimal)}
      \label{fig:IMROPT31}
    \end{subfigure}

    \begin{subfigure}{.45\textwidth}
      \centering
      %\includegraphics[width=.8\linewidth]{image_file_name}
      \caption{Infant Mortality $\geq$ 32 Weeks (Early Cohorts)}
      \label{fig:IMROPT32}
    \end{subfigure}
    \begin{subfigure}{.45\textwidth}
      \centering
      %\includegraphics[width=.8\linewidth]{image_file_name}
      \caption{Infant Mortality $\leq$ 31 Weeks (Early Cohorts)}
      \label{fig:IMROPT31}
    \end{subfigure}
    \floatfoot{Note: Each sub-plot estimates the impact of crossing the
      VLBW threshold on infant mortality.  Left-hand panels present estimates
      for gestational weeks 32 and above (where assignment rules apply), while
      right hand panels present estimates for gestational weeks 31 and below
      where assignment rules suggest no differential assignment.  Panels
      (a) and (b) replicate \citet{Bharadwajetal2013}'s methods using overlapping
      (30 g) bins and a 100 gram bandwidth.  Panels (c) and (d) use optimal
      bandwidth and bin selection.  Panels (e) and (f) replicate optimal
      plots from panels (c) and (d) focusing only on earlier birth cohorts,
      who are represented as mothers in the intergenerational sample.}
  \end{figure}


  Tabular results from above.
  

  Could also add days of hospitalization at birth here following figure 1 (alternatively, could look at this in sub-section below, or as an appendix).
  
  \subsection{Longer-Term Health Outcomes of the First Generation}
  Let's look at health outcomes over the full life, as well as their dynamic nature.  Ie first look at total nights hospitalization.  Then look at effects by age (1 yr, 2 yrs, ...). This could be shown graphically with a single point estimate and CI per year.  Then break out and look at particular ICD classes (perhaps this is conditional on there being significant impacts on health)?
  
  
  \subsection{The Intergenerational Transmission of Health at Birth}

  Should start with an illustrative bin scatter \citep{Cattaneoetal2019} showing general patterns of intergenerational transmission.

  Then should look at birth outcomes: birth weight, gestation, size.  I think we should also look at impacts across the range of birth weight (and size?) distribution.  I can implement multiple hypothesis corrections for this.

  \subsection{Identification Checks}
  Checks of covariate balance of mothers and fathers [FIRST GENERATION!!]  (ie grandparents of our most recent generation).  Density manipulations checks.  Most recent iterations \citep{Cattaneoetal2020}, earlier iterations \citet{McCrary2008}

  

  
  \clearpage
  \section{Discussion}

  \clearpage
  \end{spacing}
  \bibliography{refs}
  
  \clearpage
  \begin{appendices}
  \setcounter{page}{1}
  \renewcommand{\thepage}{A\arabic{page}}
  \setcounter{table}{0}
  \renewcommand{\thetable}{A\arabic{table}}
  \setcounter{figure}{0}
  \renewcommand{\thefigure}{A\arabic{figure}}



  \section{Data Consistency and Descriptions}
  \label{app:data}
  In progress...


  \begin{table}[htpb!]
    \centering
    \caption{Matched Observations between Microdata Registers}
    \begin{tabular}{lcccc} \toprule
      Register & Observations & Matched to & Matched to  & Matched to \\
               &              & Births     & Hospitalization & Deaths \\ \midrule
      Births          & x,xxx,xxx  & xxx,xxx    & x,xxx,xxx & xx,xxx \\
      Hospitalization & xx,xxx,xxx & xx,xxx,xxx & ---       & xx,xxx \\
      Deaths          & x,xxx,xxx  & xxx,xxx    & x,xxx,xxx & ---    \\ \midrule
      \multicolumn{5}{p{12.4cm}}{\footnotesize Notes: Column 1 presents the
        total number of observations in each dataset between 1992 and 2017.
        Column 2 notes the number of births which match to each dataset.  In the
        case of the birth register, it refers to the number of births which match
        to other births in the data (ie mother--child links). Column 3 notes the
        number of hospitalizations which match to each other database.  Note that
        in the case of births, the nuber of hospitalizations linked to births is not
        the same as the number of births linked to hospitalizations in the preceding
        column given that a single birth can be hospitalized multiple times.  Finally,
        column 4 notes the total number of deaths which are matched with births
        [AND HOSPITALIZATIONS???] occurring in the sample.}
    \end{tabular}
  \end{table}


    \begin{table}[htpb!]
    \centering
    \caption{Data Consistency Checks of Matched Microdata Bases}  
    \begin{tabular}{lccc} \toprule
               & \multicolumn{3}{c}{Linkage Register} \\ \cmidrule(r){2-4}  
      Original Register & Births & Hospitalization & Deaths \\ \midrule
      \multicolumn{4}{l}{\textbf{Panel A: Inconsistencies in Exact Birth Dates}} \\
      Births          & xx\% & xx\% & xx\% \\
      Hospitalization \hspace{2cm} & xx\% &      & xx\% \\
      Deaths          & xx\% & xx\% &      \\
      \multicolumn{4}{l}{\textbf{Panel B: Inconsistencies in Sex}} \\
      Births          & xx\% & xx\% & xx\% \\
      Hospitalization & xx\% &      & xx\% \\
      Deaths          & xx\% & xx\% &      \\ \bottomrule
      \multicolumn{4}{p{10.2cm}}{\footnotesize Each proportion refers to
        the proportion of observations which are observed with one value
        for birth date (panel A) or sex (panel B) in the ``Original Register''
        and another (inconsistent) value in the linked register. In the
        hospitalization register, prior to year xxxx exact date of birth
        is not provided, and so ``inconsistencies'' here refer only to cases
        which record different birth dates, not birth dates which are recorded
        in the birth register, and then not recorded in the hospitalization
        register.  Similarly, in a small number of cases, sex is reported
        as unkown or intersex.  Here inconsistencies only refer to cases where
        sex is recorded as female in one database, and male in another.} \\
    \end{tabular}
  \end{table}

    
  \begin{landscape}
  \begin{table}[htpb!]
    \centering
    \caption{Temporal Links between Mother--Child Matched Birth Years}
    \label{tab:birthChart}
    \begin{tabular}{lcccccccccccccccc} \toprule
    Mother & \multicolumn{16}{c}{Child} \\ \cmidrule(r){2-17}
    Year & 2003 & 2004 & 2005 & 2006 & 2007 & 2008 & 2009 & 2009 & 2010 & 2011 & 2012 & 2013 & 2014 & 2015 & 2016 & 2017 \\ \midrule
    1992 & x & xx & xxx & xxx & xxx & xxx & xxx & xxx & xxx & xxx & xxx & xxx & xxx & xxx & xxx & xxx \\
    1993 & x & xx & xxx & xxx & xxx & xxx & xxx & xxx & xxx & xxx & xxx & xxx & xxx & xxx & xxx & xxx \\
    1994 & 0 & 0  & xxx & xxx & xxx & xxx & xxx & xxx & xxx & xxx & xxx & xxx & xxx & xxx & xxx & xxx \\
    1995 & 0 & 0  & 0   & xxx & xxx & xxx & xxx & xxx & xxx & xxx & xxx & xxx & xxx & xxx & xxx & xxx \\
    1996 & 0 & 0  & 0   & xxx & xxx & xxx & xxx & xxx & xxx & xxx & xxx & xxx & xxx & xxx & xxx & xxx \\
    1997 & 0 & 0  & 0   & 0   & xxx & xxx & xxx & xxx & xxx & xxx & xxx & xxx & xxx & xxx & xxx & xxx \\
    1998 & 0 & 0  & 0   & 0   & 0   & xxx & xxx & xxx & xxx & xxx & xxx & xxx & xxx & xxx & xxx & xxx \\
    1999 & 0 & 0  & 0   & 0   & 0   & xxx & xxx & xxx & xxx & xxx & xxx & xxx & xxx & xxx & xxx & xxx \\
    2000 & 0 & 0  & 0   & 0   & 0   & xxx & xxx & xxx & xxx & xxx & xxx & xxx & xxx & xxx & xxx & xxx \\
    2001 & 0 & 0  & 0   & 0   & 0   & xxx & xxx & xxx & xxx & xxx & xxx & xxx & xxx & xxx & xxx & xxx \\
    2002 & 0 & 0  & 0   & 0   & 0   & xxx & xxx & xxx & xxx & xxx & xxx & xxx & xxx & xxx & xxx & xxx \\
    2003 & 0 & 0  & 0   & 0   & 0   & xxx & xxx & xxx & xxx & xxx & xxx & xxx & xxx & xxx & xxx & xxx \\
    2004 & 0 & 0  & 0   & 0   & 0   & xxx & xxx & xxx & xxx & xxx & xxx & xxx & xxx & xxx & xxx & xxx \\
    2005 & 0 & 0  & 0   & 0   & 0   & xxx & xxx & xxx & xxx & xxx & xxx & xxx & xxx & xxx & xxx & xxx \\
    2006 & 0 & 0  & 0   & 0   & 0   & xxx & xxx & xxx & xxx & xxx & xxx & xxx & xxx & xxx & xxx & xxx \\
    \bottomrule
    \end{tabular}
  \end{table}
  \end{landscape}


  \clearpage
  \section{Appendix Figures and Tables}
    
  \end{appendices}
\end{document}
